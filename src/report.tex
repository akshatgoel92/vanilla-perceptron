\documentclass{article}
\usepackage[utf8]{inputenc}
\usepackage{amsmath,amsthm, amssymb, pdfpages}

\title{Intro to DL Assignment 1}
\author{Orthogonals}
\date{October 2020}

\begin{document}

\maketitle
\newpage
\begin{enumerate}
    %Q1
    \item
    \newpage

    %Q2
    \item
    \newpage

    %Q3
    \item

    \begin{enumerate}
        \item

        \begin{align*}
            f(x) = \sigma(W_3\sigma(W_2\sigma(W_1 x + b1) + b2) + b3)
        \end{align*}
        We can simplify the expression to help us with derivations.
        \begin{align*}
            z_1 &= W_1 x + b_1\\
            a_1 &= \sigma(z_1)\\
            z_2 &= W_2 a_1 + b_2\\
            a_2 &= \sigma(z_2)\\
            z_3 &= W_3 a_2 + b_3\\
            f(x) &= a_3 = \sigma(z_3)
        \end{align*}
        The number of training examples can be written as m. Dimensions of all the terms are:
        \begin{align*}
            x \in\mathbb{R}^{28^2 \times m}\\
            W_1 \in\mathbb{R}^{h_1 \times 28^2}\\
            b_1 \in\mathbb{R}^{h_1 \times 1}\\
            z_1 \in\mathbb{R}^{h_1 \times m}\\
            a_1 \in\mathbb{R}^{h_1 \times m}\\
            \\
            W_2 \in\mathbb{R}^{h_2 \times h_1}\\
            b_2 \in\mathbb{R}^{h_2 \times 1}\\
            z_2 \in\mathbb{R}^{h_2 \times m}\\
            a_2 \in\mathbb{R}^{h_2 \times m}\\
            \\
            W_3 \in\mathbb{R}^{1 \times h_2}\\
            b_3 \in\mathbb{R}^{1 \times 1}\\
            z_3 \in\mathbb{R}^{1 \times m}\\
            f(x), y, a_3 \in\mathbb{R}^{1 \times m}
        \end{align*}
        Biases are broadcasted to match the activations in each layer eg. In layer 1, the bias vector will be broadcasted into a matrix of $h_1\times m$.
        \newpage
        The cost function is given by:
        \begin{align*}
            L &= -\frac{1}{m} \sum_{i=1}^{m} y_i\exp(\hat{y_i}) + (1-y_i)\exp(1-\hat{y_i})
        \end{align*}
        Where $\hat{y}$ is $a_3$, our prediction. We can also write it as a vector of losses where each element shows the loss of a single training example. Writing in this form makes our derivation simpler. We can sum over the elements to get the total loss:
        \begin{align*}
            L &= -\frac{1}{m}(y\exp(a_3) + (1-y)\exp(1-a_3))
        \end{align*}
        \begin{align*}
            \frac{d L}{d a_3} &= \frac{d}{d a_3}(-\frac{1}{m}(y\exp(a_3) + (1-y)\exp(1-a_3)))\\
            &= -\frac{1}{m} (\frac{y}{a_3} - \frac{1-y}{1-a_3})\\
            &= \frac{1}{m} \frac{a_3 - y}{a_3(1-a_3)}\\
            \\
             \frac{d L}{d z_3} &= \frac{d L}{d a_3}\frac{d a_3}{d z_3}\\
             &= \frac{d L}{d a_3} \sigma(z_3)(1 - \sigma(z_3)) \\
             &= \frac{1}{m} \frac{a_3 - y}{a_3(1-a_3)} a_3(1-a_3)\\
             &= \frac{a_3 - y}{m}\\
             \\
             \frac{d L}{d W_3} &= \frac{d L}{d z_3} \frac{d z_3}{d W_3}\\
             &= \frac{d L}{d z_3} a_2^{T}\\
             \intertext{The transpose ensures that the dimensions of the final matrix has the right dimensions and the matrix multiplication is valid.}
            \frac{d L}{d b_3} &= \frac{d L}{d z_3} \frac{d z_3}{d b_3}\\
                &= \frac{d L}{d z_3}\\
            \\
            \frac{d L}{d a_2} &= \frac{d L}{d z_3}\frac{d z_3}{d a_2}\\
            &= W_3^T \frac{d L}{d z_3}\\
            \\
            \frac{d L}{d z_2} &= \frac{d L}{d a_2}\frac{d a_2}{d z_2}\\
            &= \frac{d L}{d a_2} \sigma(z_2)(1-\sigma(z_2))\\
            &= \frac{d L}{d a_2} \circ \sigma(z_2) \circ (1-\sigma(z_2))\\
            \\
            \frac{d L}{d W_2} &= \frac{d L}{d z_2} \frac{d z_2}{d W_2}\\
            &= \frac{d L}{d z_2} a_1^{T}\\
            \\
            \frac{d L}{d b_2} &= \frac{d L}{d z_2} \frac{d z_2}{d b_2}\\
            &= \frac{d L}{d z_2}\\
            \\
            \frac{d L}{d a_1} &= \frac{d L}{d z_2}\frac{d z_2}{d a_1}\\
            &= W_2^T \frac{d L}{d z_2}\\
        \end{align*}
        \begin{align*}
            \frac{d L}{d z_1} &= \frac{d L}{d a_1}\frac{d a_1}{d z_1}\\
            &= \frac{d L}{d a_1} \sigma(z_1)(1-\sigma(z_1))\\
            &= \frac{d L}{d a_1} \circ \sigma(z_1) \circ (1-\sigma(z_1))\\
            \\
            \frac{d L}{d W_1} &= \frac{d L}{d z_1} \frac{d z_1}{d W_1}\\
            &= \frac{d L}{d z_1} x^{T}\\
            \\
            \frac{d L}{d b_1} &= \frac{d L}{d z_1} \frac{d z_1}{d b_1}\\
            &= \frac{d L}{d z_1}
        \end{align*}
        \newpage
    \end{enumerate}
    \newpage
    %Q4
    \item
    \newpage

    %Q5
    \item
    \newpage
\end{enumerate}
\end{document}
